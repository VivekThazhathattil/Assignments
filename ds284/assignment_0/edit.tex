\documentclass[12pt,letterpaper,fleqn]{article}
\usepackage[utf8]{inputenc}	
\usepackage{amsmath,amsthm,amsfonts,amssymb,amscd}
\usepackage{multirow,booktabs}
\usepackage[table]{xcolor}
\usepackage{amssymb}
\usepackage{fullpage}
\usepackage{lastpage}
\usepackage{enumitem}
\usepackage{fancyhdr}
\usepackage{mathrsfs}
\usepackage{wrapfig}
\usepackage{graphicx}
\graphicspath{ {./img/} }
\usepackage{setspace}
\usepackage{calc}
\usepackage{multicol}
\usepackage{cancel}
\usepackage[margin=3cm]{geometry}
\usepackage{floatrow}
\newlength{\tabcont}
\addtolength{\jot}{10pt}
\setlength{\parindent}{0.0in}
\setlength{\parskip}{0.05in}
\title{Assignment 0 - Vivek T, 17397}
\newcommand\course{17397}	
\newcommand\semester{2021} 
\newcommand\yourname{Vivek T}  
\theoremstyle{definition}
\usepackage{mathtools}
\DeclarePairedDelimiterX\set[1]\lbrace\rbrace{\def\given{\;\delimsize\vert\;}#1}
\newtheorem{defn}{Definición}
\newtheorem{reg}{Regla}
\newtheorem{ejer}{EJERCICIO}
\pagestyle{fancyplain}
\headheight 32pt
\lhead{\yourname\ \vspace{0.1cm} \\ \course}
\chead{\textbf{\Large DS284 Assignment-0}}
\rhead{16/08/2021}
\begin{document}
(1.a) Given:

\begin{equation}
\mathbb{V} = 
\set[\Bigg]{
\begin{pmatrix}
w\\
x\\
y\\
z\\
\end{pmatrix}
\given w - x - y + z = 0; w,x,y,z \in \mathbb{R}}
\end{equation}

Consider three vectors $u_1$, $u_2$ and $u_3$ such that $u_1, u_2, u_3 \in \mathbb{V}$.\\
Let 
\begin{equation*}
u_1 =
\begin{pmatrix}
w_1\\
x_1\\
y_1\\
z_1
\end{pmatrix}\texttt{,}~
u_2 = 
\begin{pmatrix}
w_2\\
x_2\\
y_2\\
z_2
\end{pmatrix}~\texttt{and}~
u_3 = 
\begin{pmatrix}
w_3\\
x_3\\
y_3\\
z_3
\end{pmatrix}
\end{equation*}
Checking for the conditions necessary for satisfying a real vector space,
\begin{enumerate}
\item Closure under addition property:
\begin{equation*}
\begin{split}
&u_1 + u_2 = 
\begin{pmatrix}
w_1 + w_2\\
x_1 + x_2\\
y_1 + y_2\\
z_1 + z_2
\end{pmatrix}\\
&(w_1 + w_2) -(x_1 + x_2) - (y_1 + y_2) + (z_1 + z_2)\\
&= (w_1 - x_1 - y_1 + z_1) + (w_2 - x_2 - y_2 + z_2)
\end{split}
\end{equation*}
Since $u_1,~u_2 \in \mathbb{V}$, $(w_1 - x_1 - y_1 + z_1) = 0 ~\texttt{and}~(w_2 - x_2 - y_2 + z_2) = 0$.
\begin{equation*}
\begin{split}
&\implies (w_1 - x_1 - y_1 + z_1) + (w_2 - x_2 - y_2 + z_2) = 0\\
&\implies u_1 + u_2 \in \mathbb{V}
\end{split}
\end{equation*}
Thus the closure under addition property is satisfied.\\

\item Commutative property
\begin{equation*}
\begin{split}
u_1 + u_2 &= 
\begin{pmatrix}
w_1 + w_2\\
x_1 + x_2\\
y_1 + y_2\\
z_1 + z_2
\end{pmatrix}\\
&= \begin{pmatrix}
w_2 + w_1\\
x_2 + x_1\\
y_2 + y_1\\
z_2 + z_1
\end{pmatrix} ~\because~ w,x,y,z \in \mathbb{R} \texttt{ obeys commutativity}\\
&= \begin{pmatrix}
w_2\\
x_2\\
y_2\\
z_2 
\end{pmatrix} + 
\begin{pmatrix}
w_1\\
x_1\\
y_1\\
z_1 
\end{pmatrix}\\
&= u_2 + u_1
\end{split}
\end{equation*}
Hence commutativity is satisfied.

\item Associative property
\begin{equation*}
\begin{split}
u_1 + (u_2 + u_3) &=
\begin{pmatrix}
w_1\\
x_1\\
y_1\\
z_1
\end{pmatrix} +
\begin{pmatrix}
w_2 + w_3\\
x_2 + x_3\\
y_2 + y_3\\
z_2 + z_3
\end{pmatrix}\\
&= \begin{pmatrix}
w_1 + w_2 + w_3\\
x_1 + x_2 + x_3\\
y_1 + y_2 + y_3\\
z_1 + z_2 + z_3
\end{pmatrix}\\
&= \begin{pmatrix}
w_1 + w_2\\
x_1 + x_2\\
y_1 + y_2\\
z_1 + z_2
\end{pmatrix} + 
\begin{pmatrix}
w_3\\
x_3\\
y_3\\
z_3
\end{pmatrix}\\
&= (u_1 + u_2) + u_3
\end{split}
\end{equation*}
Hence associativity is satisfied.

\item Zero vector
\begin{equation*}
\begin{split}
u_1 + 0 &= \begin{pmatrix}
w_1\\
x_1\\
y_1\\
z_1
\end{pmatrix} + 
\begin{pmatrix}
0\\
0\\
0\\
0
\end{pmatrix}\\
&= \begin{pmatrix}
w_1 + 0\\
x_1 + 0\\
y_1 + 0\\
z+1 + 0
\end{pmatrix}\\
&= \begin{pmatrix}
w_1\\
x_1\\
y_1\\
z_1
\end{pmatrix}, \because w,x,y,z \in \mathbb{R} \texttt{ obeys additive property}\\
&= u_1
\end{split}
\end{equation*}
Hence additivity property is satisfied for any $u_1 \in \mathbb{V}$

\item Additive inverse
$u_1 \in \mathbb{V}$.\\
 Then $ -1 \cdot u_1 \in \mathbb{V}$,\\
  since $ w_1 - x_1 - y_1 + z_1 = 0 \implies -1 \cdot w_1 + 1 \cdot x_1 + 1 \cdot y_1 - 1 \cdot z_1 = 0$\\
   and $ -w_1, -x_1, -y_1, -z_1 \in \mathbb{R}$.\\
   $\implies \forall u_1 \in \mathbb{V}, \exists -u_1 \in \mathbb{V}$ such that $ u_1  + (-u_1) = 0$, where $-u_1$ is the additive inverse.

\item Closure under scalar Multiplication
Let $c \in \mathbb{R}$.\\
Then,
\begin{equation*}
\begin{split}
c \cdot u_1 &= c \cdot \begin{pmatrix}
w_1\\
x_1\\
y_1\\
z_1
\end{pmatrix}\\
&= \begin{pmatrix}
c \cdot w_1\\
c \cdot x_1\\
c \cdot y_1\\
c \cdot z_1
\end{pmatrix}\\
\end{split}
\end{equation*}
$\because u_1 \in \mathbb{V}$, $w_1 - x_1 - y_1 + z_1 = 0$.\\
Multiplying both LHS and RHS by scalar constant $c$,\\
$ c \cdot (w_1 - x_1 - y_1 + z_1) = 0$.\\
$ \implies c \cdot w_1 - c \cdot x_1 - c \cdot y_1 + c \cdot z_1 = 0$\\
$ \implies c \cdot u_1 \in \mathbb{V}$.

\item Distributive property for scalar addition\\
Let $c,d \in \mathbb{R}$.\\
Then,
\begin{equation*}
\begin{split}
(c + d)\cdot u_1 &= (c + d) \cdot \begin{pmatrix}
w_1\\
x_1\\
y_1\\
z_1
\end{pmatrix}\\
&= \begin{pmatrix}
(c + d)\cdot w_1\\
(c + d)\cdot x_1\\
(c + d)\cdot y_1\\
(c + d)\cdot z_1
\end{pmatrix}\\
&= \begin{pmatrix}
c \cdot w_1 + d \cdot w_1\\
c \cdot x_1 + d \cdot x_1\\
c \cdot y_1 + d \cdot y_1\\
c \cdot z_1 + d \cdot z_1
\end{pmatrix}\\
&= \begin{pmatrix}
c \cdot w_1\\
c \cdot x_1\\
c \cdot y_1\\
c \cdot z_1
\end{pmatrix} +
\begin{pmatrix}
d \cdot w_1\\
d \cdot x_1\\
d \cdot y_1\\
d \cdot z_1
\end{pmatrix}\\
&= c \cdot \begin{pmatrix}
w_1\\
x_1\\
y_1\\
z_1
\end{pmatrix} +
d \cdot \begin{pmatrix}
w_1\\
x_1\\
y_1\\
z_1
\end{pmatrix}\\
&= c \cdot u_1 + d \cdot u_1
\end{split}
\end{equation*}
\newpage
\item Distribute property for vector addition
Let $c \in \mathbb{R}$.\\
Then,
\begin{equation*}
\begin{split}
c (u_1 + u_2) &= c \cdot \left(\begin{pmatrix}
w_1\\
x_1\\
y_1\\
z_1
\end{pmatrix} + 
\begin{pmatrix}
w_2\\
x_2\\
y_2\\
z_2
\end{pmatrix}\right)\\
&= c \cdot \begin{pmatrix}
w_1 + w_2\\
x_1 + x_2\\
y_1 + y_2\\
z_1 + z_2
\end{pmatrix} \\
&= \begin{pmatrix}
c \cdot (w_1 + w_2)\\
c \cdot (x_1 + x_2)\\
c \cdot (y_1 + y_2)\\
c \cdot (z_1 + z_2)
\end{pmatrix}\\
&= \begin{pmatrix}
c \cdot w_1 + c \cdot w_2\\
c \cdot x_1 + c \cdot x_2\\
c \cdot y_1 + c \cdot y_2\\
c \cdot z_1 + c \cdot z_2
\end{pmatrix}\\
&= \begin{pmatrix}
c \cdot w_1\\
c \cdot x_1\\
c \cdot y_1\\
c \cdot z_1
\end{pmatrix} + \begin{pmatrix}
c \cdot w_2\\
c \cdot x_2\\
c \cdot y_2\\
c \cdot z_2
\end{pmatrix}\\
&= c \cdot \begin{pmatrix}
w_1\\
x_1\\
y_1\\
z_1
\end{pmatrix} + c \cdot \begin{pmatrix}
w_2\\
x_2\\
y_2\\
z_2
\end{pmatrix}\\
&= c \cdot u_1 + c \cdot u_2
\end{split}
\end{equation*}

\item Identity operation
\begin{equation*}
\begin{split}
1 \cdot u_1 &= 1 \cdot
\begin{pmatrix}
w_1\\
x_1\\
y_1\\
z_1
\end{pmatrix}\\
&= \begin{pmatrix}
1 \cdot w_1\\
1 \cdot x_1\\
1 \cdot y_1\\
1 \cdot z_1
\end{pmatrix}\\
&= \begin{pmatrix}
w_1\\
x_1\\
y_1\\
z_1
\end{pmatrix}\\
&= u_1
\end{split}
\end{equation*}
\end{enumerate}

Hence, $\mathbb{V}$ forms a vector space as it satisfies all the conditions required for a vector space.
\newpage
(1.b) Given:
 \begin{align}
\mathbb{M}^{2\times2} = \set[\bigg]{
\begin{pmatrix}
a& 1\\
b& c\\
\end{pmatrix}
\given a,b,c \in \mathbb{R}
} 
\end{align}
Let $u_1 \in \mathbb{M}^{2 \times 2}$, such that
\begin{equation*}
\begin{split}
u_1 = \begin{pmatrix}
a_1 &1\\
b_1 &c_1
\end{pmatrix}
\end{split}
\end{equation*}
Checking for the conditions for vector space,
\begin{enumerate}
\item Additive inverse
\begin{equation*}
\begin{split}
u_1 = \begin{pmatrix}
a_1 &1\\
b_1 &c_1
\end{pmatrix}\\
\end{split}
\end{equation*}
In order to have an additive inverse, say $v \in \mathbb{M}^{2 \times 2}$, the condition $ u_1 + v = 0$ must be satisfied.\\
Then $v = -u_1$.
\begin{equation*}
-u_1 = 
\begin{pmatrix}
-a_1 &-1\\
-b_1 &-c_1
\end{pmatrix}
\end{equation*}
But $-u_1 \notin \mathbb{M}^{2 \times 2}$ as it is not of the form:
\begin{equation*}
\begin{pmatrix}
a &1\\
b &c
\end{pmatrix}
\end{equation*}
\end{enumerate}
Hence, $\mathbb{M}^{2 \times 2}$ does not form a vector space.
\newpage
 (1.c) Given:
 \begin{align*}
 \mathbb{N} = \set[\Bigg]{f: \mathbb{R} \rightarrow \mathbb{R} \given \frac{df}{dx} + 2f = 1}
 \end{align*}
 
\newpage
(2.a) Given:
	\begin{align*}
	\mathbb{V} = \set[\Bigg]{
	\begin{pmatrix}
	x\\
	y\\
	z
\end{pmatrix}	\given x,y,z \geq 0	
	}
	\end{align*}

 (b) Given:
	\begin{align*}
	\mathbb{V} = \set[\Bigg]{
	\begin{pmatrix}
	x\\
	y\\
	z
\end{pmatrix}	\given x,y,z \in \mathbb{R},~ x^2 = z^2	
	}
	\end{align*}

 (c) Given:
 	\begin{align*}
	\mathbb{V} = \set[\Bigg]{
	\begin{pmatrix}
	a& b\\
	c& d\\
\end{pmatrix}	\given \texttt{det}
	\begin{pmatrix}
	a& b\\
	c& d\\
	\end{pmatrix} = 0
	}
	\end{align*}

 (d) To prove: Intersection of two subspaces of a vector space $\mathbb{V}$ over a field $\mathbb{F}$ is a subspace of $\mathbb{V}$.

\newpage
4.
(a)
	\begin{figure}[h!]
		\includegraphics[height=10cm]{4a1}
		\includegraphics[height=10cm]{4a2}
	\end{figure}
\newpage
	\begin{figure}[h!]
		\centering
		\includegraphics[height=10cm]{4a3}	
	\end{figure}
	Given system of linear equations for each set can be represented
	as straight lines on a 2D plane. Their intersection points corresponds to the possible solutions.\\
	Set 1: Unique solution at $(x,y) = (-1, 2)$\\
	Set 2: no solution (since both lines are parallel and not coincident)\\
	Set 3: infinite solutions (since both lines are parallel and coincident)\\

\newpage
4.
(b) Each set of linear system of equations can be algebraically solved via elimination method or by using their matrix representation.\\
\\	
	\textbf{Set 1 (i) Elimination method:}
	\begin{equation*}
	\begin{split}
	&x + 2y = 3\\
	&\implies x = 3 -2y\\
	\\
	&4x + 5y = 6\\
	&\texttt{So,}~4(3- 2y) + 5y = 6\\
	&12 - 8y + 5y = 6\\
	&3y = 6 \\
	&y = 2\\
	\\
	&\implies x + 2\cdot 2 = 3\\
	&x = -1 
	\end{split}
	\end{equation*}
	\textbf{ (ii) Matrix method:}\\
	The system of equations can be written in matrix representation form as:
	\begin{equation*}
	\begin{split}
	\begin{pmatrix}
	1 &2\\
	4 &5
	\end{pmatrix}
	\begin{pmatrix}
	x\\
	y
	\end{pmatrix} =
	\begin{pmatrix}
	3\\
	6
	\end{pmatrix}
	\end{split}
	\end{equation*}
	which is of the form $\textbf{A}\cdot \textbf{x} = \textbf{b}$.
	\begin{equation*}
	\begin{split}
	\texttt{det}(\textbf{A}) &= ( 5 \cdot 1 - 2 \cdot 4 )\\
	&= -3
	\end{split}
	\end{equation*}
	det(\textbf{A}) is non zero. Hence the square matrix \textbf{A} is non-singular and $\textbf{A}^{-1}$ exists.\\
	Then $\textbf{x} = \textbf{A}^{-1}\textbf{b}$.
	
	\begin{equation*}
	\begin{split}
	\textbf{A}^{-1} &= \frac{1}{-3} \times 
	\begin{pmatrix}
	5 &-2\\
	-4 &1
	\end{pmatrix}\\
	\\
	&= \begin{pmatrix}
	-5/3 &2/3\\
	4/3 &-1/3
	\end{pmatrix}\\
	\end{split}
	\end{equation*}
	\begin{equation*}
	\begin{split}
	\textbf{x} &= \textbf{A}^{-1} \textbf{b}\\
	\\
	&= 
	\begin{pmatrix}
	-5/3 &2/3\\
	4/3 &-1/3	
	\end{pmatrix} \times 
	\begin{pmatrix}
	3\\
	6	
	\end{pmatrix}\\
	&= 
	\begin{pmatrix}
	-1\\
	2
	\end{pmatrix}
	\end{split}
	\end{equation*}
	$\therefore ~x = -1$ and $y = 2$ is the solution.
	
%%%%%%%%%%%%%%%%%%%%%%%%%%%%%%%%%%%%%%%%%%%%%
	\textbf{Set 2 (i) Elimination method:}
	\begin{equation*}
	\begin{split}
	&x + 2y = 3\\
	&\implies x = 3 -2y\\
	\\
	&4x + 8y = 6\\
	&\texttt{So,}~4(3- 2y) + 8y = 6\\
	&12 - 8y + 8y = 6\\
	&12 = 6 \\
	\end{split}
	\end{equation*}
	which is not true. Hence the system of equations has no solution.\\
	\\
	\textbf{ (ii) Matrix method:}\\
	The system of equations can be written in matrix representation form as:
	\begin{equation*}
	\begin{split}
	\begin{pmatrix}
	1 &2\\
	4 &8
	\end{pmatrix}
	\begin{pmatrix}
	x\\
	y
	\end{pmatrix} =
	\begin{pmatrix}
	3\\
	6
	\end{pmatrix}
	\end{split}
	\end{equation*}
	which is of the form $\textbf{A}\cdot \textbf{x} = \textbf{b}$.
	\begin{equation*}
	\begin{split}
	\texttt{det}(\textbf{A}) &= ( 8 \cdot 1 - 2 \cdot 4 )\\
	&= 0
	\end{split}
	\end{equation*}
	det(\textbf{A}) is zero. Hence the square matrix \textbf{A} is singular and $\textbf{A}^{-1}$ does not exist.\\
	Simplifying the system of equations,
	\begin{equation*}
	\begin{split}
	x + 2y &= 3\\
	x + 2y &= 3/2
	\end{split}
	\end{equation*}
	Subtracting first equation from the second equation, we get $0 = -3/2$. Since this is not possible for any combination of $x$ and $y$, this system of equations has no solutions.\\
%%%%%%%%%%%%%%%%%%%%%%%%%%%%%%%%%%%%%%%%%%%%%%%%%
	\textbf{Set 3 (i) Elimination method:}
	\begin{equation*}
	\begin{split}
	&x + 2y = 3\\
	&\implies x = 3 -2y\\
	\\
	&4x + 8y = 12\\
	&\texttt{So,}~4(3- 2y) + 8y = 12\\
	&12 - 8y + 8y = 12\\
	&12 = 12 \\
	\end{split}
	\end{equation*}
	which is true for infinitely many values of $x$ and $y$ which satisfies any one of the equations. Hence the system of equations has infinitely many solutions.\\
	\\
	\textbf{ (ii) Matrix method:}\\
	The system of equations can be written in matrix representation form as:
	\begin{equation*}
	\begin{split}
	\begin{pmatrix}
	1 &2\\
	4 &8
	\end{pmatrix}
	\begin{pmatrix}
	x\\
	y
	\end{pmatrix} =
	\begin{pmatrix}
	3\\
	12
	\end{pmatrix}
	\end{split}
	\end{equation*}
	which is of the form $\textbf{A}\cdot \textbf{x} = \textbf{b}$.
	\begin{equation*}
	\begin{split}
	\texttt{det}(\textbf{A}) &= ( 8 \cdot 1 - 2 \cdot 4 )\\
	&= 0
	\end{split}
	\end{equation*}
	det(\textbf{A}) is zero. Hence the square matrix \textbf{A} is singular and $\textbf{A}^{-1}$ does not exist.\\
	Simplifying the system of equations,
	\begin{equation*}
	\begin{split}
	x + 2y &= 3\\
	x + 2y &= 3
	\end{split}
	\end{equation*}
	Since both the equations in the system simplifies to the same equation, both lines have infinitely many coincident points and thus the system has infinitely many solutions.\\
%%%%%%%%%%%%%%%%%%%%%%%%%%%%%%%%%%%%%%%%%%%%%%%%%
\newpage
4.
(c) For each set of linear system of equations, of the form:
\begin{equation*}
\begin{split}
a_1x + b_1y = c_1\\
a_2x + b_2y = c_2
\end{split}
\end{equation*}
\begin{equation*}
\begin{pmatrix}
a_1 &b_1\\
a_2 &b_2
\end{pmatrix}
\begin{pmatrix}
x\\
y
\end{pmatrix} = x
\begin{pmatrix}
a_1\\
a_2
\end{pmatrix}
+ y
\begin{pmatrix}
b_1\\
b_2
\end{pmatrix} =
\begin{pmatrix}
c_1\\
c_2
\end{pmatrix}
\end{equation*}
where,
\begin{equation*}
\begin{split}
\textbf{a} = 
\begin{pmatrix}
a_1\\
a_2
\end{pmatrix} \texttt{,}~
\textbf{b} = 
\begin{pmatrix}
b_1\\
b_2
\end{pmatrix}~ \texttt{and}~
\textbf{c} = 
\begin{pmatrix}
c_1\\
c_2
\end{pmatrix}
\end{split}
\end{equation*}
This can be thought of as checking whether the vector space spanned by the basis vectors $\textbf{a}$ and $\textbf{b}$ contains the vector $\textbf{c}$. If it does contain $\textbf{c}$, then that implies that atleast a solution exists for $x$ and $y$.\\
\textbf{Set 1}\\
\begin{equation*}
\begin{pmatrix}
1 &2\\
4 &5\\
\end{pmatrix}
\begin{pmatrix}
x\\
y
\end{pmatrix} =
\begin{pmatrix}
3\\
6
\end{pmatrix}
\end{equation*}
\textbf{Set 2}\\
\begin{equation*}
\begin{pmatrix}
1 &2\\
4 &8\\
\end{pmatrix}
\begin{pmatrix}
x\\
y
\end{pmatrix} =
\begin{pmatrix}
3\\
6
\end{pmatrix}
\end{equation*}
\textbf{Set 3}\\
\begin{equation*}
\begin{pmatrix}
1 &2\\
4 &8\\
\end{pmatrix}
\begin{pmatrix}
x\\
y
\end{pmatrix} =
\begin{pmatrix}
3\\
12
\end{pmatrix}
\end{equation*}
%%%%%%%%%%%%%%%%%%%%%%%%%%%%%%%%%%%%%%%%%%%%%%%%%

\newpage
5. 
\begin{align*}
\textbf{a} = 
\begin{pmatrix}
-209/362\\
-209/362\\
209/362
\end{pmatrix},~
\textbf{b} = 
\begin{pmatrix}
0\\
-408/577\\
-408/577
\end{pmatrix},~
\textbf{c} = 
\begin{pmatrix}
396/485\\
-198/485\\
198/485
\end{pmatrix}
\end{align*}
Rewriting,
\begin{equation*}
\begin{split}
\textbf{a} = 209/362 \times
\begin{pmatrix}
-1\\
-1\\
1
\end{pmatrix},~
\textbf{b} = 408/577 \times
\begin{pmatrix}
0\\
-1\\
-1
\end{pmatrix},~
\textbf{c} = 198/485 \times
\begin{pmatrix}
2\\
-1\\
1
\end{pmatrix}
\end{split}
\end{equation*}


(a) Dot product:
\begin{equation*}
\begin{split}
\textbf{a.b} &= \sum_{i=1}^{n} a_i b_i,~ \textbf{a,b} \in \mathbb{R}^n\\
 &= (209/362) \times (408/577) \times ((-1\cdot0) + (-1\cdot-1) + (1\cdot-1))\\
&= 0\\
\textbf{b.c} &= \sum_{i=1}^{n} b_i c_i,~ \textbf{b,c} \in \mathbb{R}^n\\
 &= (408/577) \times (198/485) \times ((0\cdot2) + (-1\cdot-1) + (-1\cdot1))\\
&= 0\\
\textbf{c.a} &= \sum_{i=1}^{n} c_i a_i,~ \textbf{c,a} \in \mathbb{R}^n\\
 &= (198/485) \times (209/362) \times ((2\cdot-1) + (-1\cdot-1) + (1\cdot1))\\
&= 0
\end{split}
\end{equation*}

(b) Geometric length:
\begin{equation*}
\begin{split}
\sqrt{\textbf{a.a}} &= \sqrt{\sum_{i=1}^{n} a_i a_i},~ \textbf{a} \in \mathbb{R}^n\\
 &= \sqrt{(209/362) \times (209/362) \times ((-1\cdot-1) + (-1\cdot-1) + (1\cdot1))}\\
&= 1\\
\sqrt{\textbf{b.b}} &= \sqrt{\sum_{i=1}^{n} b_i b_i},~ \textbf{b} \in \mathbb{R}^n\\
 &= \sqrt{(408/577) \times (408/577) \times ((0\cdot0) + (-1\cdot-1) + (-1\cdot-1))}\\
&= 1\\
\sqrt{\textbf{c.c}} &= \sqrt{\sum_{i=1}^{n} c_i c_i},~ \textbf{c} \in \mathbb{R}^n\\
 &= \sqrt{(198/485) \times (198/485) \times ((2\cdot2) + (-1\cdot-1) + (1\cdot1))}\\
&= 1\\
\\
\\
\textbf{x} &= \begin{pmatrix}
2\\
-40/57\\
8/77
\end{pmatrix}\\
\\
\sqrt{\textbf{x.x}} &= \sqrt{\sum_{i=1}^{n} x_i x_i},~ \textbf{x} \in \mathbb{R}^n\\
 &= \sqrt{(2\cdot2) + (-40/57 \cdot -40/57) + (8/77 \cdot 8/77)}\\
&\approx \sqrt{4 + 1600/3249 + 64/5929}\\
&\approx 2.12
\end{split}
\end{equation*}

\newpage
(c) 
\begin{equation*}
\begin{split}
\textbf{A} &= 
\begin{pmatrix}
-209/362 &0 &396/485\\
-209/362 &-408/577 &-198/485\\
209/362 &-408/577 &198/485
\end{pmatrix}\\
\\
\textbf{Ax} &= 
\begin{pmatrix}
-209/362 &0 &396/485\\
-209/362 &-408/577 &-198/485\\
209/362 &-408/577 &198/485
\end{pmatrix}
\times 
\begin{pmatrix}
2\\
-40/57\\
8/77
\end{pmatrix}\\
\\
&= \begin{pmatrix}
-1.070\\
-0.701\\
1.693
\end{pmatrix}\\
\\
\sqrt{\textbf{Ax}\cdot\textbf{Ax}} &= \sqrt{ (-1.070\cdot-1.070) + (-0.701\cdot-0.701) + (1.693\cdot1.693)}\\
&\approx \sqrt{1.145 + 0.491 + 2.866}\\
&\approx 2.12
\end{split}
\end{equation*}
It is observed that both $\sqrt{\textbf{Ax.Ax}}$ and $\sqrt{\textbf{x.x}}$ are equal.\\
\bf{Reason:}

\newpage
(d) 
\begin{equation*}
\begin{split}
\textbf{A}^{T}\textbf{A} &=
\begin{pmatrix}
-209/362 &-209/362 &209/362\\
0 &-408/577 &-408/577\\
396/485 &-198/485 &198/485
\end{pmatrix} \times
\begin{pmatrix}
-209/362 &0 &396/485\\
-209/362 &-408/577 &-198/485\\
209/362 &-408/577 &198/485
\end{pmatrix}\\
\\
&\approx \begin{pmatrix}
1 &0 &0\\
0 &1 &0\\
0 &0 &1
\end{pmatrix}\\
\\
\textbf{A}\textbf{A}^{T} &=
\begin{pmatrix}
-209/362 &0 &396/485\\
-209/362 &-408/577 &-198/485\\
209/362 &-408/577 &198/485
\end{pmatrix} \times
\begin{pmatrix}
-209/362 &-209/362 &209/362\\
0 &-408/577 &-408/577\\
396/485 &-198/485 &198/485
\end{pmatrix}\\
\\
&\approx \begin{pmatrix}
1 &0 &0\\
0 &1 &0\\
0 &0 &1
\end{pmatrix}
\end{split}
\end{equation*}
\end{document}